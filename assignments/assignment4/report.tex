\documentclass[a4paper,11pt]{article}
\usepackage{amssymb}
\usepackage{booktabs}
\usepackage{geometry}
\usepackage{color}
\usepackage{hyperref}
\usepackage{listings}
\usepackage{graphicx}
\usepackage{float}
\usepackage{caption}
\usepackage{subcaption}
\usepackage[T1]{fontenc}


\graphicspath{{img/}}

\setlength\parindent{0cm}

\geometry{
	includeheadfoot,
	margin=2.54cm
}

\title{
	2IL76 Algorithms for Geographic Data Set 4 \\
}
\author{
	Tim van Dalen (0744839)
	\and
	Bram Kohl (0746107)
	\and
	Bart van Wezel (0740608)
}
\date{\today}

\begin{document}
	\maketitle
	
\section*{Exercise 3}
For every region (partly or fully) within the necklace, we place a point on the necklace, as close as possible to the center of the (full) region. Then we combine the points of the regions of the same type by calculating a point between them, which is weighted by the area the region takes within the necklace. The final circle on the necklace is placed as close as possible to this point.

Example: there are 3 regions of type X which have a part in the necklace. The 3 points on the necklace closest to the centers of these regions are (0,0). 


\begin{figure}[H]
	\centering
	\def\svgwidth{0.5\textwidth}
%	\input{img/disk.pdf_tex}
	\caption{Each pair of nodes is close enough, but no disk can be made}
	\label{fig:nodisk}
\end{figure}
\end{document}
