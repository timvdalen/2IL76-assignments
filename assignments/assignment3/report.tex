\documentclass[a4paper,11pt]{article}
\usepackage{amssymb}
\usepackage{booktabs}
\usepackage{geometry}
\usepackage{color}
\usepackage{hyperref}
\usepackage{listings}
\usepackage{graphicx}
\usepackage{float}
\usepackage{caption}
\usepackage{subcaption}
\usepackage[T1]{fontenc}
\usepackage{algo}


\graphicspath{{img/}}

\setlength\parindent{0cm}

\geometry{
	includeheadfoot,
	margin=2.54cm
}

\title{
	2IL76 Algorithms for Geographic Data Set 3 \\
}
\author{
	Tim van Dalen (0744839)
	\and
	Bram Kohl (0746107)
	\and
	Bart van Wezel (0740608)
}
\date{\today}

\begin{document}
	\maketitle
	
\section*{Exercise 3}
We are given a graph G with vertices $v_1 \dots v_n$, and have to create trajectories such that there is a k-clique in G, if and only if a subset of those trajectories forms a k-group. In order to realize this, we create $n$ trajectories in the following way:\\
For $i = 1$ to $n$ 
\begin{enumerate}
	\item Place the $i^{th}$ point of trajectory $A_i$ at coordinate $(i, 0)$.
	\item For each $1 \leq j \leq n$ where $j \neq i$, if $v_i$ and $v_j$ are connected in G, place the $i^{th}$ point of trajectory $A_j$ at coordinate $(i,\varepsilon)$. Otherwise ($v_i$ and $v_j$ are not connected), place it at coordinate $(i,2\varepsilon)$.
\end{enumerate}
This way, at time $t_i$, trajectory $A_i$ and $A_j$ (for any $j$ with $1 \leq j \leq n$ and $j \neq i$) are close enough to be in a k-group together, if and only if $v_i$ and $v_j$ share an edge in G.
\end{document}
