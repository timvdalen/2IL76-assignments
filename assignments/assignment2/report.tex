\documentclass[a4paper,11pt]{article}
\usepackage{amssymb}
\usepackage{booktabs}
\usepackage{geometry}
\usepackage{color}
\usepackage{hyperref}
\usepackage{listings}
\usepackage{graphicx}
\usepackage{float}
\usepackage{caption}
\usepackage{subcaption}
\usepackage[T1]{fontenc}
\usepackage{algo}


\graphicspath{{img/}}

\setlength\parindent{0cm}

\geometry{
	includeheadfoot,
	margin=2.54cm
}

\title{
	2IL76 Algorithms for Geographic Data Set 2 \\
}
\author{
	Tim van Dalen (0744839)
	\and
	Bram Kohl (0746107)
	\and
	Bart van Wezel (0740608)
}
\date{\today}

\begin{document}
	\maketitle
	
\section*{Exercise 3}
\subsection*{Algorithm}
The algorithm we came up with is a 3 dimensional variant on the Imai-Iri algorithm. Essentially it works the same as \textsc{Imai-Iri}, but the \textsc{ValidShortcut} is different. The input for our algorithm, \textsc{Imai-Iri-3D}, is A, which is the trajectory and $\varepsilon$ which is the maximum distance of the trajectory to its simplification.

\begin{algorithm}{Imai-Iri3D}{(A,$\varepsilon$)}
 Find all possible valid shortcuts \\
Build a graph G containing all valid shortcuts. \\
Find a minimum link path from $a_{1}$ to $a_{n}$  in G using breadth-first search.
      \end{algorithm} 
      
      \begin{algorithm}{ValidShortcut}{$(A, A_{i},A_{j},\varepsilon$)}
      \qfor $k = i+1, k < j$\\
       \qif evaluate$(A_{k},A_{i},A_{j}) > \varepsilon$ \\ 
       \qreturn False \qfi \qrof \\
       \qreturn True
      \end{algorithm} 
      

In this algorithm, d(x,y) is the Manhattan distance of x to y, and evaluate$(A_k,A_i,A_j)$ is the value of the $A_i,A_j$ line on the $z$ (time) of $A_k$. This is a 3d point, but, obviously, has the same z-value as $A_k$.

\subsection*{Proof of correctness}
Our algorithm is based on \textsc{Imai-Iri}, which is obviously correct. So what is left to prove is that our adaptation does not falsify the correctness. We still check the distance of the intermediate points (the points between the endpoints of the shortcut)

\subsection*{Running time}

The running time of our algorithm is $O(n^{3})$. 
We have $O(n^{2})$ shortcuts. 
We need $O(n)$ time for each shortcut. 
We need $O(n^{2})$ time to build the graph from the shortcuts and $O(n^{2})$ time to use bread first search to find the shortest path. 
This means that the total running time of our algorithm is $O(n^{3})$, just like the original Imai-Iri algorithm. 
However the algorithm should run in $O(n^{2})$ time. 
This means that we have to speed up our algorithm.
In the lecture they speed up Imai-Iri by testing all shortcuts from a single vertex in linear time instead of in $O(n)$ time. 
You can do this by checking if both rays (from $p_{i}$ through $p_{j}$ and from $p_{j}$ through $p_{i}$) intersect all $\varepsilon$ -disks of vertices inbetween. This is possible to do in linear time when you compute the plane of these points and look if the ray is in that plane. The plane you compute in linear time, because you only add one other wedge to the plane each time when you look one more point away. 
However we cannot directly apply this method, because we look at the Manhattan distance between two points. Also our algorithm also uses time, so we would have to do it in 3D. 
We can do this by checking if the rays (from $p_{i}$ through $p_{j}$ and from $p_{j}$ through $p_{i}$) intersect the square defined by the four points: 
$(x_{k}+\varepsilon,y_{k}+\varepsilon,t_{k}), (x_{k}+\varepsilon,y_{k}-\varepsilon,t_{k}), (x_{k}-\varepsilon,y_{k}+\varepsilon,t_{k})$ and $(x_{k}-\varepsilon,y_{k}-\varepsilon,t_{k})$ for each k such that $i<k<j$. We can now maintain a pyramid. You start with the pyramid of $p_{k-1}$ and  $(x_{k}+\varepsilon,y_{k}+\varepsilon,t_{k}), (x_{k}+\varepsilon,y_{k}-\varepsilon,t_{k}), (x_{k}-\varepsilon,y_{k}+\varepsilon,t_{k})$ and $(x_{k}-\varepsilon,y_{k}-\varepsilon,t_{k})$. 
Then for each point after that till n you take the intersetion of the previous pyramid and the new pyramid. 
Then a shortcut is possible if the ray between the points is in the pyramid. 

\end{document}
