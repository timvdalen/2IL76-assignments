We assume the following two simple helper algorithms:

\begin{description}
	\item[\textsc{PerpDistance}($p$, $(q_i, q_j)$)] calculates the perpendicular distance from point $p$ to line segment $(q_i, q_j)$.
		If there is no such distance (because the perpendicular line from $p$ to the line that $(q_i, q_j)$ lies on does not cross $(q_i, q_j)$), this algorithm returns $\infty$.
	\item[\textsc{PointDistance}($p$, $q$)] calculates the Euclidean distance between points $p$ and $q$.
	\item[\textsc{LineSegLength}($(q_i, q_j)$)] calculates the length of a line segment
	\item[\textsc{LineLength}($Q$)] calculates the sum of the lengths of all line segments in a line
\end{description}

\begin{sourcecode}
\algorithm{Distance($p$, $(q_i, q_j)$}
\return{$\min\left(\textsc{PerpDistance}(p, (q_i, q_j)), \textsc{PointDistance}(p, q_i), \textsc{PointDistance}(p, q_j)\right)$}
\qend

\algorithm{SegmentMetric($(q_i, q_j)$, $P$, $Q$)}
$D$ is a map from point to distance to that point
\foreach{point $p$ in $P$}
	$D[p]$ = \textsc{Distance}($p$, $(q_i, q_j)$)
|
\return{the sum of the $\lceil |P| \frac{\textsc{LineSegLength}((q_i, q_j))}{\textsc{LineLength}(Q)} \rceil$ lowest distances in $D$}
\qend

\algorithm{PointCurveDistanceMetric($P$, $Q$)}
$\phi = 0$
\foreach{line segment $(q_i, q_j)$ in $Q$}
	$\phi = \phi + $ \textsc{SegmentMetric}($(q_i, q_j)$, $P$, $Q$)
|
\return{\phi}
\qend
\end{sourcecode}
