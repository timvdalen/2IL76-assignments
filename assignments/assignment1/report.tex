\documentclass[a4paper,11pt]{article}
\usepackage{amssymb}
\usepackage{booktabs}
\usepackage{geometry}
\usepackage{hyperref}
\usepackage{listings}
\usepackage{graphicx}
\usepackage{float}
\usepackage{caption}
\usepackage{subcaption}
\usepackage[T1]{fontenc}

\catcode`\^^I=13
\catcode`\^^M=13

\newcounter{linenumber}

\newenvironment{sourcecode}{\begingroup%
\setcounter{linenumber}{0}%
\def\algorithm##1{\setcounter{linenumber}{0}\medskip \textbf{Algorithm} \textsc{##1} \parskip=0pt\everypar={\parskip=0pt\stepcounter{linenumber}\rlap{\scriptsize{\thelinenumber}}\qquad}}%
\def\event##1{\setcounter{linenumber}{0}\medskip \textbf{##1:} \parskip=0pt\everypar={\parskip=0pt\stepcounter{linenumber}\rlap{\scriptsize{\thelinenumber}}\qquad}}%
%
\def\comment##1{\textit{//##1}}%
\def\qif##1{\textbf{if} ##1}%
\def\qifdefined##1{\textbf{if} ##1 \textbf{defined}}%
\def\qifndefined##1{\textbf{if} ##1 \textbf{not defined}}%
\def\qelse{\textbf{else}}%
\def\qelseif##1{\textbf{else if ##1}}%
\def\for##1{\textbf{for} ##1}%
\def\foreach##1{\textbf{for each} ##1}%
\def\qdo{\textbf{do}}%
\def\while##1{\textbf{while} ##1}%
\def\return##1{\textbf{return} ##1}%
\def\break{\textbf{break}}%
\def\qglobal##1{\textbf{global} ##1}%
\def\qsend##1{\textbf{send} \emph{##1}}%
\def\qreceive##1{\textbf{receive} \emph{##1}}%
\def\qbegin{\textbf{begin}}
\def\qend{\everypar={}\textbf{end}}
\def\runningtime##1{\hfill ##1}
\catcode`\^^I=13%
\catcode`\^^M=13%
\def^^I{\qquad}%
\def^^M{\par}%
}{\endgroup}

\catcode`\^^I=10
\catcode`\^^M=5


\graphicspath{{img/}}

\setlength\parindent{0cm}

\geometry{
	includeheadfoot,
	margin=2.54cm
}

\title{
	2IL76 Algorithms for Geographic Data Set 1 \\
}
\author{
	Tim van Dalen (0744839)
	\and
	Bram Kohl (0746107)
	\and
	Bart van Wezel (0740608)
}
\date{\today}

\begin{document}
	\maketitle
	
\section*{Exercise 3}
We came up with a measure for how close a polygonal curve Q is to a point set P.

We define the distance of a point in P to a line segment in Q as the minimum of the perpendicular distance to the line segment and the Euclidian distances to the two endpoints of the line segment.
We define the perpendicular distance to the line segment as the perpendicular distance to the line through the endpoints of the line segments, or infinite if the intersection point of the perpendicular line with the line through the endpoints is not on the line segment.
See for example Figure~\ref{fig:perp-distance}.

Here, the perpendicular distance is infinite as $Q_I$ is not on the line segment ($Q_1,Q_2$).

\begin{figure}
	\label{fig:perp-distance}
	\def\svgwidth{0.5\textwidth}
	\input{img/perp-distance.pdf_tex}
	\caption{Infinite perpendicular distance}
\end{figure}

As not all points have to be close, we came up with a formula that defines the number of nearest points in P we will be looking at for each line segment.
We do this in order to reduce noise from outliers.
The formula we came up with is: $n\cdot \frac{l}{tl}$ where $n = |P|$, $l$ is the length of the line segment in Q we are currently looking at and $tl$ is the total length of the line segments in Q.
This way, longer lines need more close points than shorter lines.
Note that when adding the number of nearest points for each line segment you get $n$, but this does not mean all points have to be one of the nearest points of a line segment, as a point can be a nearest point for multiple line segments.

We wanted large line segments to take into consideration more nearest neighbours than small line segments because this is useful when P is well distributed over Q.
See for example figure \ref{fig:line-weight}. %TODO: formule invullen en zeggen dat lange lijn veel punten nodig heeft etc 

\begin{figure}
	\label{fig:line-weight}
	\def\svgwidth{0.5\textwidth}
	\input{img/line-width.pdf_tex}
	\caption{A point set P that is well distributed over curve Q}
\end{figure}

 
	We assume the following two simple helper algorithms:

\begin{description}
	\item[\textsc{PerpDistance}($p$, $(q_i, q_j)$)] calculates the perpendicular distance from point $p$ to line segment $(q_i, q_j)$.
		If there is no such distance (because the perpendicular line from $p$ to the line that $(q_i, q_j)$ lies on does not cross $(q_i, q_j)$), this algorithm returns $\infty$.
	\item[\textsc{PointDistance}($p$, $q$)] calculates the Euclidean distance between points $p$ and $q$.
	\item[\textsc{LineSegLength}($(q_i, q_j)$)] calculates the length of a line segment
	\item[\textsc{LineLength}($Q$)] calculates the sum of the lengths of all line segments in a line
\end{description}

\begin{sourcecode}
\algorithm{Distance($p$, $(q_i, q_j)$}
\return{$\min\left(\textsc{PerpDistance}(p, (q_i, q_j)), \textsc{PointDistance}(p, q_i), \textsc{PointDistance}(p, q_j)\right)$}
\qend

\algorithm{SegmentMetric($(q_i, q_j)$, $P$, $Q$)}
$D$ is a map from point to distance to that point
\foreach{point $p$ in $P$}
	$D[p]$ = \textsc{Distance}($p$, $(q_i, q_j)$)
|
\return{the sum of the $\lceil |P| \frac{\textsc{LineSegLength}((q_i, q_j))}{\textsc{LineLength}(Q)} \rceil$ lowest distances in $D$}
\qend

\algorithm{PointCurveDistanceMetric($P$, $Q$)}
$\phi = 0$
\foreach{line segment $(q_i, q_j)$ in $Q$}
	$\phi = \phi + $ \textsc{SegmentMetric}($(q_i, q_j)$, $P$, $Q$)
|
\return{\phi}
\qend
\end{sourcecode}

\end{document}
