\documentclass[a4paper,11pt]{article}
\usepackage{amssymb}
\usepackage{booktabs}
\usepackage{geometry}
\usepackage{hyperref}
\usepackage{listings}
\usepackage{graphicx}
\usepackage{float}
\usepackage{caption}
\usepackage{subcaption}
\usepackage[T1]{fontenc}

\catcode`\^^I=13
\catcode`\^^M=13

\newcounter{linenumber}

\newenvironment{sourcecode}{\begingroup%
\setcounter{linenumber}{0}%
\def\algorithm##1{\setcounter{linenumber}{0}\medskip \textbf{Algorithm} \textsc{##1} \parskip=0pt\everypar={\parskip=0pt\stepcounter{linenumber}\rlap{\scriptsize{\thelinenumber}}\qquad}}%
\def\event##1{\setcounter{linenumber}{0}\medskip \textbf{##1:} \parskip=0pt\everypar={\parskip=0pt\stepcounter{linenumber}\rlap{\scriptsize{\thelinenumber}}\qquad}}%
%
\def\comment##1{\textit{//##1}}%
\def\qif##1{\textbf{if} ##1}%
\def\qifdefined##1{\textbf{if} ##1 \textbf{defined}}%
\def\qifndefined##1{\textbf{if} ##1 \textbf{not defined}}%
\def\qelse{\textbf{else}}%
\def\qelseif##1{\textbf{else if ##1}}%
\def\for##1{\textbf{for} ##1}%
\def\foreach##1{\textbf{for each} ##1}%
\def\qdo{\textbf{do}}%
\def\while##1{\textbf{while} ##1}%
\def\return##1{\textbf{return} ##1}%
\def\break{\textbf{break}}%
\def\qglobal##1{\textbf{global} ##1}%
\def\qsend##1{\textbf{send} \emph{##1}}%
\def\qreceive##1{\textbf{receive} \emph{##1}}%
\def\qbegin{\textbf{begin}}
\def\qend{\everypar={}\textbf{end}}
\def\runningtime##1{\hfill ##1}
\catcode`\^^I=13%
\catcode`\^^M=13%
\def^^I{\qquad}%
\def^^M{\par}%
}{\endgroup}

\catcode`\^^I=10
\catcode`\^^M=5


\graphicspath{{img/}}

\setlength\parindent{0cm}

\geometry{
	includeheadfoot,
	margin=2.54cm
}

\title{
	2IL76 Algorithms for Geographic Data Set 1 \\
}
\author{
	Tim van Dalen (0744839)
	\and
	Bram Kohl (0746107)
	\and
	Bart van Wezel (0740608)
}
\date{\today}

\begin{document}
	\maketitle
	
\section*{Exercise 3}
We came up with a measure for how close a polygonal curve Q is to a point set P.\\

We define the distance of a point $P_1$ in P to a line segment $Q_1$ in Q as \\%tims definitie + uitleg

As not all points have to be close, we came up with a formula that defines the number of nearest points in P we will be looking at for each line segment. We do this in order to reduce noise from outliers. The formula we came up with is: $|P|\cdot \frac{l}{tl}$ where $|P|$ is the number of points in P, $l$ is the length of the line segment in Q we are currently looking at and $tl$ is the total length of the line segments in Q.\\

We wanted large line segments to take into consideration more nearest neighbours than small line segments. This is useful when P is well distributed over Q. See for example figure \ref{fig:line-weight}. Here, 

%\begin{figure}
%	\label{fig:line-weight}
%	\includegraphics{blaaaa}
%	\caption{A point set P that is well distributed over curve Q}
%\end{figure}

 
	We assume the following two simple helper algorithms:

\begin{description}
	\item[\textsc{PerpDistance}($p$, $(q_i, q_j)$)] calculates the perpendicular distance from point $p$ to line segment $(q_i, q_j)$.
		If there is no such distance (because the perpendicular line from $p$ to the line that $(q_i, q_j)$ lies on does not cross $(q_i, q_j)$), this algorithm returns $\infty$.
	\item[\textsc{PointDistance}($p$, $q$)] calculates the Euclidean distance between points $p$ and $q$.
	\item[\textsc{LineSegLength}($(q_i, q_j)$)] calculates the length of a line segment
	\item[\textsc{LineLength}($Q$)] calculates the sum of the lengths of all line segments in a line
\end{description}

\begin{sourcecode}
\algorithm{Distance($p$, $(q_i, q_j)$}
\return{$\min\left(\textsc{PerpDistance}(p, (q_i, q_j)), \textsc{PointDistance}(p, q_i), \textsc{PointDistance}(p, q_j)\right)$}
\qend

\algorithm{SegmentMetric($(q_i, q_j)$, $P$, $Q$)}
$D$ is a map from point to distance to that point
\foreach{point $p$ in $P$}
	$D[p]$ = \textsc{Distance}($p$, $(q_i, q_j)$)
|
\return{the sum of the $\lceil |P| \frac{\textsc{LineSegLength}((q_i, q_j))}{\textsc{LineLength}(Q)} \rceil$ lowest distances in $D$}
\qend

\algorithm{PointCurveDistanceMetric($P$, $Q$)}
$\phi = 0$
\foreach{line segment $(q_i, q_j)$ in $Q$}
	$\phi = \phi + $ \textsc{SegmentMetric}($(q_i, q_j)$, $P$, $Q$)
|
\return{\phi}
\qend
\end{sourcecode}

\end{document}
