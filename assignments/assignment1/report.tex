\documentclass[a4paper,11pt]{article}
\usepackage{amssymb}
\usepackage{booktabs}
\usepackage{geometry}
\usepackage{hyperref}
\usepackage{listings}
\usepackage{graphicx}
\usepackage{float}
\usepackage{caption}
\usepackage{subcaption}
\usepackage[T1]{fontenc}

\input{../algoritme}

\graphicspath{{img/}}

\setlength\parindent{0cm}

\geometry{
	includeheadfoot,
	margin=2.54cm
}

\title{
	2IL76 Algorithms for Geographic Data Set 1 \\
}
\author{
	Tim van Dalen (0744839)
	\and
	Bram Kohl (0746107)
	\and
	Bart van Wezel (0740608)
}
\date{\today}

\begin{document}
	\maketitle
	
\section*{Exercise 3}
We came up with a measure for how close a polygonal curve Q is to a point set P.\\

We define the distance of a point $P_1$ in P to a line segment $Q_1$ in Q as \\%tims definitie + uitleg

As not all points have to be close, we came up with a formula that defines the number of nearest points in P we will be looking at for each line segment. We do this in order to reduce noise from outliers. The formula we came up with is: $|P|\cdot \frac{l}{tl}$ where $|P|$ is the number of points in P, $l$ is the length of the line segment in Q we are currently looking at and $tl$ is the total length of the line segments in Q.\\

We wanted large line segments to take into consideration more nearest neighbours than small line segments. This is useful when P is well distributed over Q. See for example figure \ref{fig:line-weight}. Here, 

%\begin{figure}
%	\label{fig:line-weight}
%	\includegraphics{blaaaa}
%	\caption{A point set P that is well distributed over curve Q}
%\end{figure}

 
	We assume the following two simple helper algorithms:

\begin{description}
	\item[\textsc{PerpDistance}($p$, $(q_i, q_j)$)] calculates the perpendicular distance from point $p$ to line segment $(q_i, q_j)$.
		If there is no such distance (because the perpendicular line from $p$ to the line that $(q_i, q_j)$ lies on does not cross $(q_i, q_j)$), this algorithm returns $\infty$.
	\item[\textsc{PointDistance}($p$, $q$)] calculates the Euclidean distance between points $p$ and $q$.
\end{description}

\begin{sourcecode}
\algorithm{Distance($p$, $(q_i, q_j)$}
\return{$\min\left(\textsc{PerpDistance}(p, (q_i, q_j)), \textsc{PointDistance}(p, q_i), \textsc{PointDistance}(p, q_j)\right)$}
\qend
\end{sourcecode}

\end{document}
