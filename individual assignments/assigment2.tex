\documentclass[a4paper,11pt]{article}
\usepackage{amssymb}
\usepackage{booktabs}
\usepackage{geometry}
\usepackage{color}
\usepackage{hyperref}
\usepackage{listings}
\usepackage{graphicx}
\usepackage{float}
\usepackage{caption}
\usepackage{subcaption}
\usepackage[T1]{fontenc}
\usepackage{algo}

\geometry{
	includeheadfoot,
	margin=2.54cm
}

\setlength\parindent{24pt}

\title{
	2IL76 Algorithms for Geographic Data Set 2 \\
}
\author{
	Bram Kohl - b.j.e.kohl@student.tue.nl - 0746107
}
\date{\today}

\begin{document}
	\maketitle
	
\section*{Exercise 1}
We solve this exercise using a variant of the Agarwal et al. algorithm from the Simplification lecture. In the algorithm, we use \textsc{EnclosingDisk(P)} as a subroutine which computes the smallest enclosing disk for the set P of $k$ points. This subroutine is given in the assignment and uses O($k$) time.\\
%TODO: another difference is that we now set i to j+1 instead of j, as we only need P_j to be close enough to one circle and no longer need to connect this point to the next one (rather than just find a point which is close enough to the next one). Also leq n instead of <
\subsection*{Algorithm}
\begin{algorithm}{NonRestrictedSimplification}{(A,$\varepsilon$)}
	i := 1\\
	B = $\varnothing$\\
	\qwhile i $\leq$ n \\
	find \textit{any} j>i such that \\
	\hspace{4ex} (the radius of \textsc{EnclosingDisk($A_i\dots A_j$)} $\leq \varepsilon$ and\\
	\hspace{4ex} the radius of \textsc{EnclosingDisk($A_i\dots A_{j+1}$)} $> \varepsilon$) or\\
	\hspace{4ex} (the radius of \textsc{EnclosingDisk($A_i\dots A_j$)} $\leq \varepsilon$ and j=n)\\
	Add the center of the \textsc{EnclosingDisk($A_i\dots A_j$)} to B\\
	i := j + 1 \qend \\
	\qreturn B
\end{algorithm} 
In this algorithm, n is the length of A.
\subsection*{Proof of correctness}
In order to prove that the algorithm is correct, we have to prove two things. Firstly that the discrete Fr\'{e}chet distance between A and B is at most $\varepsilon$ and secondly that B is minimum-length.\\

For the first part, take an arbitrary point $A_x$ in A (so $1 \leq x \leq n$). In the algorithm, i is initialized on 1, after which a j is chosen for which all points from $A_i$ up to and including $A_j$ are within \textsc{EnclosingDisk($A_i$\dots $A_j$)} (we assume this function to be correct as it is given). The radius of this disk is at most $\varepsilon$. Then the center of this disk is added to the result sequence. So suppose $i \leq x \leq j$, then $A_x$ is within $\varepsilon$ of the center of this disk.\\
After this iteration of the while loop, i is set to j+1 and this continues (starting at i = 1) until i is greater than n, so at some point (as $1 \leq x \leq n$) x will be between i and j and thus $A_x$ will be within $\varepsilon$ of the center of a disk, and thus a point of B. So an arbitrary point of A is close enough to B, thus it holds for all points of A, thus the discrete Fr\'{e}chet distance of A to B is at most $\varepsilon$.\\

The second part can be proven as follows. As we use the Fr\'{e}chet distance, we cannot go backwards in time, e.g. we cannot have a point x in B associated with a point in y A that is in between two other points in A, let's say y' and y'', where y' and y'' are associated with a point x' in B for which x' $\neq$ x. Due to this, we can do a greedy approach starting at $A_1$ and associating as many points of A as possible to the same point in B, after which we can continue at the next point in A which has not been associated to a point in B yet and do the same. We need to link this first unassociated point to the upcoming point in B every time, as we cannot go back in time to include it later. Furthermore we cannot skip a point for the same reason, so we add as many points as possible without skipping any points. This way we define points in B for every point in A which include the maximum number of possible points, and thus A will be covered by a minimal number of points in B.

\subsection*{Running time}
Just like the algorithm of Agarwal et al. in the Simplification slides, we can reduce the number of iterations of the inner loop as we just need \textit{any} point for which the condition on lines 5-7 of the algorithm holds. So here, we can do the combination of Exponential search and Binary search as well. So we test whether the radius of \textsc{EnclosingDisk($A_i$\dots $A_j$)} $> \varepsilon$ for j = i+1, i+2, i+4, etc until we find a j = i + $2^k$ such that the radius of \textsc{EnclosingDisk($A_i$\dots $A_j$)} $> \varepsilon$ holds. When we find this, we do a binary search between i+$2^{k-1}$ and i+$2^k$ to find the j for which the radius of \textsc{EnclosingDisk($A_i\dots A_j$)} $\leq \varepsilon$ and the radius of \textsc{EnclosingDisk($A_i\dots A_{j+1}$)} $> \varepsilon$ (or j = n). As we do \textsc{EnclosingDisk} on $j-i$ points every time, we get O($j-i$) as running time for this. This is the same as the running time of computing the Fr\'{e}chet distance in the Simplification slides. So we get the same running time for the algorithm as the Agarwal et al. algorithm from the slides: O($n \log n$).

\section*{Exercise 2}
First, we have to prove that the above-below relation is acyclic for a monotone map M.\\
Proof: We prove this in 2 parts: first we prove that for two paths $c_i$ and $c_j$, $c_i$ is not both above and below $c_j$. This is the case as we are dealing with a monotone map. In order for $c_i$ to be both above and below $c_j$, it needs to have an endpoint above $c_j$ and one below it (or symmetrically, $c_j$ needs to have an endpoint above and one below $c_i$). So the x coordinates of both of the endpoints of $c_i$ have to be between the x coordinates of the endpoints of $c_j$, while one of the endpoints of $c_i$ is below $c_j$ and the other is above $c_j$. To accomplish this, $c_i$ would have to either cross through $c_j$, which contradicts with the definition of a map, or $c_i$ would have to go around one of the endpoints of $c_j$, which contradicts with the map being monotone.\\

In the second case, there are more than two paths involved. Let $c_i$, $c_j$ and $c_k$ be paths with $c_i$ above $c_j$ and $c_j$ above $c_k$. Now we have to prove that $c_k$ is not above $c_i$. Suppose $c_k$ is above $c_i$, then at least one of $c_k$'s endpoints, let's call it $a$, is above $c_i$. So the x coordinate of $a$ is between the x coordinates of the endpoints of $c_i$, and $a$ is above $c_i$. In order for $c_k$ to be below $c_j$, one of the endpoints of $c_k$ has to be below $c_j$ and its x coordinate needs to be in between the x coordinates of the endpoints of $c_j$. Lets call this point $b$. Then $b$ cannot reach $a$ without either 1. $c_k$ intersecting with $c_i$ or $c_j$ (which conflicts with the map definition), or 2. $c_k$ going around one of the endpoints of $c_i$ or $c_j$ (which conflicts with the monotonicity definition). As $c_j$ is below $c_i$, one of its endpoints is below $c_i$, so at least a part of the area below $c_j$ is below $c_i$ as well. If $b$ is within this part, then we have the same as in the first case of this assignment, so 1. and 2. follow. If $b$ is not below $c_i$, then it is possible it can connect to $a$ without intersecting $c_i$ and without violating the monotonicity constraint. But then (if it does not violate the monotonicity constraint), it would intersect with $c_j$, as $b$ is in the part that is underneath $c_j$ but not underneath $c_i$, and $a$ is above $c_i$ and between the x coordinates of the endpoints of $c_i$.\\
For example, look at figure \ref{fig:ass2}. Here $a$ would be in the area above $c_i$ and $b$ in the area below $c_j$. If $b$ is in the area that is below both $c_i$ and $c_j$, it cannot reach any point above $c_i$ without breaking map or monotonicity definition on $c_i$. If $b$ is in the part below $c_j$ that is not below $c_i$, then it cannot reach any point above $c_j$ without breaking the monotonicity constraint and without crossing either $c_i$ or $c_j$.\\

A total order for a monotone map can be computed efficiently using a Plane Sweep algorithm from the second slide set of the 2IL55 course (which I did not take). \url{http://www.win.tue.nl/~kbuchin/teaching/2IL55/slides/02line-segment-intersection.pdf}.\\
To do this, you sweep from left to right and when you are at the endpoint of a path you at it to your order.


\begin{figure}[H]
	\centering
	\includegraphics[scale=0.1]{assignment2.jpg}
	\caption{Infinite perpendicular distance}
	\label{fig:ass2}
\end{figure}
\end{document} 